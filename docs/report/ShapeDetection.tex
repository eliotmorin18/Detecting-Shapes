\documentclass{article}

% Language setting
% Replace `english' with e.g. `spanish' to change the document language
\usepackage[english]{babel}
\usepackage{tikz}
\usetikzlibrary{calc}
\usetikzlibrary{shapes.geometric}

% Set page size and margins
% Replace `letterpaper' with `a4paper' for UK/EU standard size
\usepackage[letterpaper,top=2cm,bottom=2cm,left=3cm,right=3cm,marginparwidth=1.75cm]{geometry}

% Useful packages
\usepackage{amsmath, amssymb}
\usepackage{graphicx} % For including images
\usepackage{subcaption} % For subfigures
\usepackage[colorlinks=true, allcolors=blue]{hyperref}


% Julia setup 
\usepackage{listings}
\usepackage{algorithm}
\usepackage{algpseudocode}
\usepackage{xcolor}

\lstdefinelanguage{Julia}{
  keywords={function, end, return, if, else, for, while, in, using, import},
  sensitive=true,
  comment=[l]\#,
  morestring=[b]",
}

\lstset{
  language=Julia,
  basicstyle=\ttfamily\small,
  keywordstyle=\color{blue},
  commentstyle=\color{gray},
  stringstyle=\color{orange},
  showstringspaces=false
}

\renewcommand\thesubsubsection{\thesubsection.\alph{subsubsection}}

\title{Computation Topology - Shape Detection}
\author{Eske Bockelmann, Elliot Morin}

\begin{document}
\maketitle


\begin{algorithm}
\caption{rips\_complex(S,r)}\label{alg:rips_complex}
\begin{algorithmic}
    \Require Point cloud $S$, radius $r$
    \Ensure Rips complex $RC$

    \State $n\gets length(S)$
    \State $RC \gets [(i,) \text{ for }i\in\, 1:n]$
    
    \For{$i \in 1:n$}
        \For{$j \in (i+1):n$}
            \If{$\|S[i] - S[j]\| \leq r$}
                \State $RC \gets RC \cup \{(i,j)\}$
            \EndIf 
        \EndFor 
    \EndFor

    \State$dim \gets 2 $
    \State$L \gets \text{All simplices in RC of length} == dim$
    \While{$L \not = \emptyset$}
        \State$dim \gets dim + 1$
        \State$new\_simplices \gets \text{all possible } dim \text{ combinations of simplices in} L$
        
        \For{sx in new\_simplices}
            \State$sx \gets \{\text{vertices of sx}\}$
        \EndFor 
        
        \State $new\_simplices \gets \text{Only keep those simplices which have exactly }dim\text{ different vertices}$
        
        \State $RC \gets RC \cup new\_simplices$
        \State $L \gets \text{All simplices in RC of length} == dim$
    \EndWhile 

    \Return $RC$
\end{algorithmic}
\end{algorithm}

\begin{algorithm}
\begin{algorithmic}
    \caption{rips\_filtration(S,R)}\label{alg:rips_filtration}
    \Require Point cloud $S$, list of different radii $R$
    \Ensure Rips filtration $RF$

    \For{$r \in R$}
        \State$RF[r] => rips\_complex(S,r)$
    \EndFor
    
    \Return $RF$
\end{algorithmic}
\end{algorithm}
\begin{algorithm}
\caption{minimal\_enclosing\_ball(S) = MEB(S)} \label{alg:minimal_enclosing_ball}
\begin{algorithmic}
    \Require Point cloud $S$
    \Ensure radius and origin of minimal enclosing ball $r, \text{origin}$

    \State$n \gets length(S)$
    \If{n == 2}
        \State$r \gets \|S[1] - S[2] \|/2$
        \State$\text{origin} = S[1] + \frac{1}{2}*(S[2] - S[1])$ 
        
        \State \Return $r, \text{origin}$
    \Else 
        \For{$i \in 1:n$}
            \State$r,\text{origin} \gets minimal\_enclosing\_ball(S \setminus S[i])$
            \If{$\|S[i] - \text{origin} \| \leq r $} \Comment{S[i] in the MEB$(S\setminus S[i])$}
                
                \State \Return $r,\text{origin}$
            \EndIf
        \EndFor
        
        \Comment{Otherwise the MEB is the ball through all vertices}
        \State$d \gets length(S[1])$
        
        \State$ A = 2 * \left(\begin{aligned}
             (S[2] &- S[1])^T \\
             &\vdots \\
             (S[n] &- S[1])^T 
            \end{aligned}\right) \in \mathbb{R}^{n-1,d}$

        \State$ b =\left(\begin{aligned}
            \| S[2] \| ^2 &- \| S[1] \| ^2 \\
            &\vdots \\ 
            \| S[n] \| ^2 &- \| S[1] \| ^2 
            \end{aligned}\right) \in \mathbb{R}^{n-1}$
        \State$\text{origin} = A\setminus b$
        \State$r = \| \text{origin } - S[1] \| $
        \State \Return $r,\text{origin}$
    \EndIf
\end{algorithmic}
\end{algorithm}

\begin{algorithm}
\caption{cech\_complex(S,r)}\label{alg:cech_complex}
\begin{algorithmic}
    \Require Point cloud $S$, radius $r$
    \Ensure Cech Complex $CC$ 

    \State $n\gets length(S)$
    \State $CC \gets [(i,) \text{ for }i\in\, 1:n]$
    
    \For{$i \in 1:n$}
        \For{$j \in (i+1):n$}
            \If{$\|S[i] - S[j]\| \leq r$}
                \State $CC \gets CC \cup \{(i,j)\}$
            \EndIf 
        \EndFor 
    \EndFor

    \State$dim \gets 2 $
    \State$L \gets \text{All simplices in RC of length} == dim$
    \While{$L \not = \emptyset$}
        \State$dim \gets dim + 1$
        \State$new\_simplices \gets \text{all possible } dim \text{ combinations of simplices in} L$
        
        \For{sx in new\_simplices}
            \State$sx \gets \{\text{vertices of sx}\}$
        \EndFor 
        
        \State $new\_simplices \gets \text{Only keep those simplices which have exactly }dim\text{ different vertices}$
        \State $new\_simplices \gets \text{Only keep those simplices where the radius of MEB(vertices) }\leq r$
        
        
        \State $CC \gets CC \cup new\_simplices$
        \State $L \gets \text{All simplices in RC of length} == dim$
    \EndWhile 

    \State \Return $CC$

\end{algorithmic}
\end{algorithm}

\begin{algorithm}
\begin{algorithmic}
    \caption{cech\_filtration(S,R)}\label{alg:cech_filtration}
    \Require Point cloud $S$, list of different radii $R$
    \Ensure Cech filtration $CF$

    \For{$r \in R$}
        \State$CF[r] => cech\_complex(S,r)$
    \EndFor
    
    \Return $CF$
\end{algorithmic}
\end{algorithm}


\begin{lstlisting}
    include("graphcomponents.jl")
    # example 1 from excercise    
    V = [1,2,3,4,5,6,7,8]
    E = [(1,2),(2,3),(1,3),(4,5),(5,6),(5,7),(6,7),(7,8)]
    # example to write some julia code in latex with nice highlighting
\end{lstlisting}
%

\end{document}

